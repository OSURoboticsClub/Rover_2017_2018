\subsection{Ken Steinfeldt}
\subsubsection{Technical Knowledge Gained}
I gained a lot of technical knowledge and experience from the project.
First and foremost I learned how to use frameworks that I had never experienced before in ROS and PyQT5.
Not only had I never used Qt before, but I had never worked with any real UI framework before this project.
The ROS framework was also a unique challenge.
This project was an example of a very real use case for ROS, doing exactly what it was designed to do. 
The framework was more complicated than I had anticipated, and required a significant learning curve. 
Using both ROS and PyQt5 on a project such as this required using the frameworks according to best practice, and was fairly complicated at first, but simpler as time went on.
\subsubsection{Non-Technical Knowledge Gained}
I think that the most important non-technical knowledge I gained was the experience of working on a large, complicated project. 
As the ground station was only a small portion of the of the Mars Rover project, it relied on many modules in many different locations.
As a group we were forced to manage these problems as they arose. 
Even the organization of the repository was a challenge and had to be redone multiple times.
Development was also a problem as it could be a challenge to set up a functional development and testing environment. 
This was alleviated with the development of a startup script that Corwin wrote. 
\subsubsection{Knowledge Gained of Project Work}
Software engineering is notorious for delays and missed deadlines, and I am beginning to understand why.
At the beginning of the project many modules seemed to have a clear path to completion and simple solutions.
We thought that it would be easy for us to learn the two frameworks and then just start working with them, after all, that's what frameworks are for. 
As it turns out, both ROS and Qt have not insignificant learning curves. 
Learning how to use these frameworks took me much longer than I thought it would, and that hampered me throughout the process.
I also learned to trust my teammates, which is not something that someone expects to learn from a group project. 
My teammates were great at always came through, even when I was not always able to. 
By the end of the project I had learned that I could trust them completely.
\subsubsection{Knowledge Gained of Project Management}
The biggest thing that I learned about project management was how much easier it is to do when the project is well defined and planned at the beginning. 
The work that we were required to do at the beginning of the year defined the scope of the project, the goals, the problems, and their planned solutions. 
Having this already done due to the requirements of the class was hugely beneficial in implementing our solutions and a lesson in project management for the future.
\subsubsection{Knowledge Gained of Working in Teams}
Our group had great communication which was very beneficial to the team.
This open communication helped to make clear what was expected of each group member and helped our group to have fun and get along.
\subsubsection{Changes to Make if Starting Over}
\begin{itemize}
\item Take more time to understand the frameworks before trying to code with them.
\item Spend more time with the group personally.
\item Spend more time working with the rover team on rover stuff. 
\item Take the time to create the testing environment before starting.
\end{itemize}