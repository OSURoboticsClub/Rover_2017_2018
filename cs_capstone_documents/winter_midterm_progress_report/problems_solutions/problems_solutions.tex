\section{Problems / Solutions}
\subsection{Mars Rover Electrical/Software Team Development Progress}
As the Mars Rover project is a very large multi-disciplinary project run by undergraduate volunteers, the team has historically had problems with being behind schedule, and this year is no different. The team lost their electrical lead last term and has been catching up since the team lead, Nick McComb, took over this position. This, plus normal delays in development have meant that many of the core systems that the capstone team needs to test against are not present, not ready to interact with in their current state, or in a few cases still in a research state. Corwin has attempted to alleviate some of this by helping assemble hardware for the electrical team and prioritizing some ground station development time for the Rover systems themselves. If need be, our capstone team could help with these other systems more if need be to help facilitate forward progress.


\subsection{ROS / QT Learning Curves}
The Robot Operating System and QT frameworks that are at the core of the ground station project are large, involved tools that require extensive practice and training to effectively use. As these are relatively new technologies to even the main Rover software team, the learning curve has slowed initial development. Overall, we don't think this will continue to be a problem as the team gets more and more familiar with them, but it definitely impacted development time up to this point.
